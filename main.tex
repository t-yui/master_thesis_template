\documentclass[10.5pt,a4j]{ujreport}
\usepackage[top=2cm, bottom=2cm, left=2cm, right=2cm]{geometry}
\usepackage[dvipdfmx]{graphicx}
%---------------------------------------------------
\usepackage{algorithmic}
\usepackage{algorithm}
\usepackage{bm}
%---------------------------------------------------
% chapterで改ページしない
\makeatletter
\renewcommand{\chapter}{%
  \if@openright\cleardoublepage\fi
  \thispagestyle{jpl@in}%
  \global\@topnum\z@
  \@afterindenttrue
  \secdef\@chapter\@schapter}
\makeatother
%---------------------------------------------------
% eqnarrayで等号の両端のスペースを小さくする
\setlength\arraycolsep{1.5pt}
%---------------------------------------------------

%---------------------------------------------------
% 表紙
%---------------------------------------------------
\title{
  {\Huge
    修 士 学 位 論 文
  }
  \\[3cm]
  {\Huge
    〇〇についての研究
  }
  \\[5cm]
}

\author{
  {\huge
    氏名
  }\\[0.5cm]
  東京大学 大学院学際情報学府\\
  学際情報学専攻 生物統計情報学コース\\
  49-186608\\[0.5cm]
  指導教員 〇〇〇〇 教授
}
\date{\Large 令和元年度}


\begin{document}
\maketitle


%---------------------------------------------------
% 抄録
%---------------------------------------------------
\chapter*{抄録}
抄録を記述する.


%---------------------------------------------------
% 序論
%---------------------------------------------------
\chapter{序論}
\pagenumbering{arabic}

\section{背景}
背景を記述する.


\section{目的}
目的を記述する.


\chapter{方法}
方法を記述する.


\chapter{結果と考察}
結果と考察を記述する.


\chapter{結論}
結論を記述する.


%---------------------------------------------------
% 謝辞
%---------------------------------------------------
\chapter*{謝辞}
\addcontentsline{toc}{chapter}{謝辞}
本研究を行うに当たり,多大なるご理解とご指導を頂きました〇〇教授に心より御礼申し上げます.


%---------------------------------------------------
% 参考文献
%---------------------------------------------------
\renewcommand{\bibname}{参考文献}
\addcontentsline{toc}{chapter}{参考文献}
\bibliography{bibliography} %hoge.bibから拡張子を外した名前
\bibliographystyle{unsrt} %参考文献出力スタイル


%---------------------------------------------------
% 図表
%---------------------------------------------------
\chapter*{図表(通し番号とする)}
\addcontentsline{toc}{chapter}{図表(通し番号とする)}

% 図・表を通し番号にする
\makeatletter
\def\delete#1from#2{%
\def\reserved@a{#1}%
\def\reserved@c{}%
\def\@elt##1{%
\def\reserved@b{##1}%
\ifx\reserved@a\reserved@b
\else
\@cons\reserved@c{{##1}}%
\fi}%
\csname cl@#2\endcsname
\expandafter\let\csname cl@#2\endcsname\reserved@c
\let\@elt\relax}
\makeatother
\delete{figure}from{chapter}
\renewcommand*{\thefigure}{\arabic{figure}}
\delete{table}from{chapter}
\renewcommand*{\thetable}{\arabic{table}}


図表を記述する.


\end{document}